\chapter*{Conclusion}

In this dissertation I have demonstrated three axis along which it is possible to align the molecular phenotype of gene expression. In Chapter \ref{chpt:plasticity}, we align the molecular phenotypes of \textit{Saccharomyces cerevisiae} and \textit{Saccharomyces paradoxus} by a set of environmental perturbations and measure how clusters of genes change expression in each species in response to the environmental change. By aligning on this axis, we find that  most expression plasticity is conserved, but each environment reveals a different subset of genes diverging in plasticity. This plasticity divergence was primarily attributable to divergence in the cellular environment in response to certain environments. These changes to the cellular environment affected the expression of hundreds of genes in \textit{trans}. We saw that genes known to be involved in each environmental response in \textit{S. cerevisiae} tended to have conserved plasticity in that environment. In other environments, however, these genes did not necessarily have conserved plasticity. This suggests that plasticity divergence may be caused by aberrant expression of genes that do not play a role in that particular environmental response but do have a role in other environments.

In Chapter \ref{chpt:networks}, we align by a set of orthologous transcription factors that are deleted in both species and measure how each deletion affects each species' ability to respond to a signal of nitrogen repletion. We first recapitulated two of the findings of Chapter \ref{chpt:plasticity} using different methods: 1) Expression plasticity in response to low nitrogen tended to be conserved between wildtype genetic backgrounds. 2) These conserved groups of genes have the strongest enrichment of functional gene ontology groups. We then compared the degree of conservation of the wildtype response to conservation among 46 transcription factor knockout lines and found a very different pattern. Here our results agreed with those of several recent papers. Specifically that the vast majority of genes that were differentially expressed in a given TF knockout line in one genetic background were not differentially expressed in another background. In the context of the low nitrogen response, we were able to describe a common pattern to these TF knockout effects despite their affecting different genes. Specifcally that the knockout lines tended to lessen the expression responsiveness during the low nitrogen response compared with wildtype yeast.

In Chapter \ref{chpt:misexpression}, we align by a set of genes that are misexpressed in the F1 hybrid of \textit{Saccharomyces cerevisiae} and \textit{Saccharomyces paradoxus} and begin to measure how different combinations of parental alleles influences this misexpression in F2 hybrids. We find that transcriptomic data alone is sufficient to identify and define recombination in F2 hybrids. Our preliminary analysis of these F2 transcriptomes further suggests that overexpression and underexpression in a hybrid genome may be caused by different types of molecular interactions. This is because we find that genes underexpressed in F1s tended to be overexpressed in F2s even when compared to non-misexpressed genes of similar expression level and different F2 haploid genomes.

What do we gain from considering these three chapters as alignments along axes as opposed to three separate studies that ask three separate questions? A common challenge is faced whenever we ask questions about gene expression or other molecular dynamics that don't pertain to specific biological systems. When analyzing large biological datasets in multiple species, any direction the researcher chooses to interrogate the data, differences between species can be identified. The challenge is not finding these differences so much as it is contextualizing them. I argue that the key to interpreting these differences between species is to put them in context with how much conservation and divergence tends to be exhibited across molecular systems. The conceptual framework of an alignment guarantees that this sort of context is given. By first aligning data based on an aspect of shared biology rather than by any aspect of divergence or conservation that may be known beforehand about these species, and second characterizing the extent of conservation and divergence, the researcher captures the tendencies of how molecular systems are diverging in context with how much is staying the same.