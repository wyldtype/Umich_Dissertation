\chapter{Comparison of signal propagation through the low nitrogen response network in two yeast species}
\label{chpt:networks}

\section{Introduction}
%\label{sec:intro}

Signal propagation through gene regulatory networks is a vital process through which organisms respond to changes in their environment. Signaling begins when environment-sensing proteins, henceforth called \hl{``sensors''}, alter their structure or function due to a change in cellular state, triggering a gene regulatory cascade \cite{Bahn2007}. This cascade ultimately results in the upregulation of genes whose products play important roles in adjusting to the environmental change, henceforth called \hl{``responders''}. For example, \hl{add PKA/TOR sensor to this example} upregulation of enzymes involved in breaking down alternative carbon sources in response to a glucose shortage \cite{Gancedo1998}. While protein function can be remarkably conserved over long evolutionary timescales, recent studies have found surprising variation in how environmental signals transduce from sensors to responders, even between closely related species. For example, \hl{Cite comparative signaling, innate immunity?}. This variation has important implications for how regulatory networks evolve. In particular, how much neutral space---variation of little phenotypic consequence---is available in signaling networks structure for populations to explore over evolutionary time \cite{Payne2015}.

$<$ Most TFdel effects are 1) different for different genetic backgrounds (even differing by one present/absent gene), 2) not in genes that the TF binds directly to, 3) Conserved effects remain conserved farther out in evo time (these could be the direct effects, but that hasn't been tested yet) $>$

The intermediaries in a signal transduction response are transcription factors (TFs). A common strategy to assess how genes are connected in a regulatory network is to examine which genes have altered mRNA abundance in response to repression of specific TFs. Recent studies in yeast have taken advantage of gene editing tools and a relatively small set of TFs (169 recorded in the Yeastract database \cite{Hahn2011}) to compare how the same set of TF knockouts affect gene expression in different genetic backgrounds. Surprisingly, these studies have found that the majority of differentially expressed genes in each TF knockout line are non-overlapping sets. In fact, the sets of differentially expressed genes between \textit{S. cerevisiae} and \textit{S. pombe} in response to an orthologous gene deletion have as much overlap as the gene sets from a randomly selected pair of non-orthologous gene deletions \cite{Li2025}. The majority of effects tend to be measured in genes located far from where the TF bound, suggesting these are largely pleiotropic \cite{Mahendrawada2025}. The minority \hl{percent} of genes with conserved effects between strains tend to remain conserved between more distantly related species, but it is unclear whether these effects are more likely to have evidence of direct TF binding \cite{Liu2024}. Taken together, these studies suggest that this lack of predictability in which genes will be differentially expressed in response to a given TF perturbation is the result of widespread pleiotropic expression responses that change rapidly with evolutionary time. Within a specific genotype, however, these effects are intriguingly stable. The identity, magnitude, and direction of differentially expressed genes are highly repeatable in multiple measurements of the same genetic background and can be detected within \hl{X minutes} of the onset of TF repression \cite{Mahendrawada2025}.

$<$ What is causing these pleietropic effects if not TF binding or global response? $>$

At present, there is little mechanistic understanding of why certain genes will display a pleiotropic response to a given TF perturbation. Cells that are missing a highly connected regulator are expected to be in a stressed state. The pleiotropic effects may be a general response to this stress state, such as DNA replication stress \hl{cite DNA replication stress}. Different genetic backgrounds may exhibit different generalized stress responses, resulting in significant non-overlap between differentially expressed genes. A single, generalized stress response, however, cannot explain all the observed effects of TF knockouts. This is because the set of differentially expressed genes in response to a given TF knockout are highly dependent on the identity of the TF \cite{Liu2024}. These TF-specific effects suggest that the probability of a gene being differentially expressed upon TF perturbation is related to its position in the regulatory network relative to the perturbed TF. Determining what aspects of network position are most predictive of regulatory relationships is an open-ended question. One strategy is to evaluate the effects of TF knockouts in the context of a well studied signaling pathway, where the regulators are known and response genes can be measured. The effects of TF knockouts can be assessed by asking a) whether the deleted TF is a known component of the signaling pathway or not, and b) how the differentially expressed genes in that TF knockout are related to the set that responds during wildtype signaling. This type of analysis requires knowledge of a well characterized signaling pathway and time series expression data of a response to the induced signal.

%Connecting between the TF and a given target gene could be as simple as one degree of separation, for example via a secondary regulator that in turn directly regulates the target gene, or it might require multiple degrees of separation. Connections in the network may represent DNA binding interactions, as when a TF binds to the \textit{cis}-regulatory region of the target gene. More complicated connections are also possible, such as when a co-factor is recruited to bind the DNA or the TF indirectly influences a direct regulator of the affected gene via a change in cellular state.

%Systematically testing this possibility would require complete knowledge of every possible regulatory interaction and the environmental conditions in which that interaction can take place. In lieu of complete knowledge, any instance where there a regulatory relationship cannot be drawn between the perturbed TF and a given differentially expressed gene would either be the result of a true lack of interaction or a lack of data, and therefore inconclusive. 


$<$ The Low Nitrogen pathway $>$

The Low Nitrogen response pathway is an extensively studied environmental response pathway in \textit{S. cerevisiae} \cite{Magasanik2005}. The low nitrogen signal begins when a depletion of cellular nitrogen triggers the protein Ure2p to form a prion aggregate, [URE3], and fall out of solution \cite{Wickner1994}. Soluble Ure2p binds the GATA-factor TFs Gat1p and Gln3p thereby sequestering them in the cytosol. When the aggregation of [URE3] is triggered, Gat1p and Gln3p are released and imported into the nucleus to trigger a regulatory cascade \cite{Zhang2018}. Gat1p and Gln3p work in complex with themselves and with the other zinc-finger TFs Dal80p and Gzf3p to promote expression of enzymes to break down alternative nitrogen sources. Much of what is known about this pathway has been studied in \textit{S. cerevisiae}. The sister species \textit{S. paradoxus} also has a low nitrogen response, but less is known about the molecular machinery producing this response \hl{S. paradoxus LowN response}. These species additionally form F1 hybrids, allowing the assessment of the low nitrogen response in a mixed regulatory environment. 

Here we evaluate the effects of 46 TF knockouts in \textit{S. cerevisiae}, \textit{S. paradoxus}, and their F1 hybrid in the context of the low nitrogen response. We first identify the sets of genes in each of the three wildtype genotypes that alter expression in response to a low nitrogen signal. We find the largest category of nitrogen-responsive genes were conserved between the three genotypes, but these made up a minority of all gene responsive genes (32 \%). \hl{This set was enriched/was not enriched for nitrogen catabolism GO terms compared to the non-conserved sets}. While all three genotypes had sets of genes that were unique, we found that the hybrid regulatory environment could recapitulate nearly every gene expression response that was shared between the two parental species, with the exception of only 8 out of 404 genes. For responses that were not conserved, the hybrid regulatory environment showed strong dominance for a single parental species' regulatory response shape, but which parent was dominant depended on the gene. When we examined how 46 TF knockouts affected each genotype's low nitrogen response, we indeed found that different knockouts affected different genes and that only 6\% were conserved effects between all three genotypes. The effects of the knockouts had a common theme of reducing the variability of nitrogen-responding genes, \hl{More or less true for the conserved LowN response set}. Taken together, our results demonstrate that network connectivity enhances the ability of genes to respond to the low nitrogen environment.

\section{Results}
%\label{sec:resultsanddiscussion}

To establish a frame of reference for our TF knockout comparisons, we first characterized the low nitrogen response in wildtype \textit{S. cerevisiae}, \textit{S. paradoxus}, and their F1 hybrid.

\subsection{}