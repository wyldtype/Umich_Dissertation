The molecular systems that underly similar organismal phenotypes can vary to a surprising degree between closely related species. In this dissertation, I explore the ways these molecular systems tend to diverge across the genome. I focus on comparing one pair of species, the \textit{Saccharomyces} yeasts  \textit{Saccharomyces cerevisiae} and \textit{Saccharomyces paradoxus}, and one way of measuring the molecular environment, comparing gene expression. To compare what aspects of gene expression are conserved and which are diverged between these two species, I align their molecular phenotype of gene expression along three different axes. In Chapter \ref{chpt:plasticity}, I align along the axis of gene expression responses to six different changes in environment. In Chapter \ref{chpt:networks}, I align along the axis of deletions of 46 orthologous transcription factors. In Chapter \ref{chpt:misexpression}, I align along the axis of a set of genes that are overexpressed or underexpressed in F1 hybrids of the two species. These alignments allow me to contextualize the myriad of differences in molecular systems observed between these species by always comparing what has diverged to what remains conserved between species.